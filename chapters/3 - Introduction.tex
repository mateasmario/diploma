\chapter{Introduction}
\thispagestyle{pagestyle}

If we go back in time and review the history of computers, we can observe that various programming techniques have been adopted over the years. Computer programming has undergone one of the most significant evolutionary processes, beginning almost a century ago when instructions were sent to machines through plugboards and switches. Several decades later, computer engineers transitioned to punch card programming and also witnessed the release of the famous COBOL language. Some of these individuals may still be alive today, reminiscing about the times when development was not as simple as it is today.

The significant improvements from the 1930s to the present day were influenced by two major factors:

\begin{enumerate}
  \item the emergence of higher-level languages (e.g. C, FORTRAN) and languages that were built for specific purposes (e.g. Java, that redefined the way object-oriented programs were written);
  \item the constant release of tools that aided the programmer and simplified the entire development process (so-called "integrated development environments") \cite{ide_bigbang}.
\end{enumerate}

Due to the increased difficulty of maintaining (or migrating) legacy code, such as banking applications still relying on COBOL, modern tools have been developed to aid programmers even under such conditions. This means that if refactoring (or a "hotfix") is required, programmers won't need to manipulate punch cards but can simply use their keyboard and mouse, as they are already accustomed to. This situation confirms that, in several cases, integrated development environments play a more significant role than the actual programming language in the software development process. If we consider popular IDEs such as Microsoft's Visual Studio Code \cite{vscode}, we can also confirm that users are indeed able to write code even in a language they aren't familiar with, thanks to the integration of artificial intelligence.

Integrated development environments should provide every functionality needed by a software developer while working on a project. However, there are several cases where such a tool provides more than required, and using it may become a burden for the user \cite{ide_productivity}. IDEs were built to make the entire coding process more time-efficient, but they do exactly the opposite when the developer isn't accustomed to their features or when using such a tool for a small-scaled application. In the same manner, moving on to a "simpler" editor, such as Notepad++, which focuses primarily on text editing, might also not be a good solution, as it may not provide all the necessary tools for development.

With this in mind, I have implemented Pie: a lightweight code editor, useful for scripting or coding small projects that do not require the feature bloat available in today's popular IDEs. Pie provides text editing capabilities, similar to Notepad++, but it also integrates features most commonly required during application development. These capabilities include: code syntax highlighting and autocomplete, text formatting options, visualization of data available via the persistence layer, local repository management, and the ability to store and access custom build commands. Additionally, Pie provides real-time rendering of HyperText Markup Language (HTML) and Markdown code. All of these features can be accessed through a docked menu strip, where they are classified into categories and subcategories. Apart from the top menu, Pie's interface displays only the developer's main focus area - either the code editor, the output of a rendered HTML page, or the current status of the local git repository - with no other controls that might distract the user displayed unless explicitly toggled. 

As previously mentioned, the project is a general-purpose code editor, providing support for the most commonly used programming languages today. These include, but are not limited to: C, C\#, Java, JavaScript, JSON, Lua, Python, XML, HTML, and SQL. Pie has also been intended to be fully customizable. In addition to saving commands and storing database connection information, users can completely alter its color combinations. I have included several default themes that I believe look appealing; however, users are free to modify these or extend the list by adding their own custom themes.

Pie is a free software solution, designed as an open-source product from its first day of implementation. It is the result of more than a year of intense work and refinement, although it was not initially intended for use as a diploma project. I aimed to create a product that would assist me in situations where an IDE was just too much. Throughout the implementation process, I used Pie as my primary code and text editor, completely replacing Notepad and Notepad++. Whenever I realized that I needed a quicker way to perform a task, such as removing duplicate lines in text, and anticipated using this feature more than once, I would close everything and begin developing that feature.

The rest of the paper will be structured as follows: Chapter 4 presents similar products that were used as sources of inspiration for Pie, detailing differences and similarities between my product and state-of-the-art IDEs. Architectural aspects are considered in Chapter 5, along with the frameworks and third-party dependencies I have used. This chapter will also present the structure of the application, emphasizing the five main components that make up Pie. The implementation of Pie is discussed in Chapter 6, where I will present several code snippets that show how the user interface logic is separated from the business logic of the application. A demonstration of the project will be displayed in Chapter 7, while we conclude the paper and mention future work in Chapter 8.