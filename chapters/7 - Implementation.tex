\chapter{Implementation}
\thispagestyle{pagestyle}

This section will cover the implementation of Pie's key features, including how to handle configuration files, apply color palettes to WinForms controls, manage tabs in the code editor and terminal, enable syntax highlighting for different languages, handle database connections, use LibGit2Sharp for repository management, and apply text formatting algorithms. I will also include code snippets, application screenshots and detailed explanations to ensure everything is understandable.

\section{JSON-based configuration in Pie}

User-defined configuration is stored in JSON files within the \texttt{config} directory, located relative to the executable's root folder. The main window of the application, defined in file \texttt{MainForm.cs}, loads this configuration data in the constructor. This means the processing occurs when the application first becomes visible on the screen.

The code below shows a part of the \texttt{ProcessBuildCommands()} method, which is first called in constructor of the \texttt{MainForm} class.

\begin{lstlisting}[language=csharp, caption={Build command processing logic in MainForm's constructor}]
try
{
    Globals.buildCommands = BuildCommandService.GetBuildCommandsFromFile("config/build.json");

    // Add processed build commands to the top menu strip
    (...)
}
catch (FileNotFoundException ex)
{
    // If the file does not exist, create a new one
    File.WriteAllText(AppDomain.CurrentDomain.BaseDirectory + "config/build.json", "[]");
}

\end{lstlisting}

Commands will be extracted from the \texttt{config/build.json} file only if the file exists. If the file has been accidentally deleted, an empty JSON file will be created to prevent future errors. This approach to error management is more user-friendly than displaying an error message about a potential missing file. Such messages might prompt users to reinstall the product, as they may not be familiar with the structure of the missing file. The commands are saved in a global \texttt{List} that can be accessed by multiple classes (such as forms and services).

The logic of the \texttt{BuildCommandService.GetBuildCommandsFromFile()} method is presented in the snippet below:

\begin{lstlisting}[language=csharp, caption={Reading build commands saved in the configuration file through the BuildCommandService}]
public static List<BuildCommand> GetBuildCommandsFromFile(string file)
{
    List<BuildCommand> buildCommands = new List<BuildCommand>();
    BuildCommand buildCommand = null;
    string token = null;

    // Open the configuration file
    using (var textReader = File.OpenText(AppDomain.CurrentDomain.BaseDirectory + file))
    {
        // Instantiate the NewtonsoftJson Reader
        JsonTextReader jsonTextReader = new JsonTextReader(textReader);

        // Repeat until the entire file has been read
        while (jsonTextReader.Read())
        {
            if (jsonTextReader.Value != null)
            {
                // Map data to BuildCommand objects
                if (jsonTextReader.TokenType == JsonToken.PropertyName)
                {
                    token = jsonTextReader.Value.ToString();
                }
                else if (jsonTextReader.TokenType == JsonToken.String)
                {
                    if (token == "name")
                    {
                        buildCommand = new BuildCommand();
                        buildCommand.BuildCommandName = jsonTextReader.Value.ToString();
                    }
                    else if (token == "command")
                    {
                        buildCommand.BuildCommandCmd = jsonTextReader.Value.ToString();
                        buildCommands.Add(buildCommand);
                    }
                }
            }
        }
    }

    return buildCommands;
}
\end{lstlisting}

The \texttt{JsonTextReader} class is part of the Newtonsoft.Json library. According to the logic above, the service only processes tokens labeled \texttt{name} and \texttt{command}, as build commands consist of user-defined names and the corresponding commands to execute. The values from each group are stored in \texttt{BuildCommand} objects, which have a field defined for each token.

An example of a \texttt{build.json} file can be found below:

\begin{lstlisting}[language=json, caption={Structure of the JSON-formatted build commands file in Pie}]
[
  {
    "name": "gcc build",
    "command": "gcc -Wall -o output $FILE"
  },
  {
    "name": "java compile",
    "command": "javac $FILE"
  },
  {
    "name": "java run main class",
    "command": "java Main"
  }
]
\end{lstlisting}

When commands are executed from Pie's user interface, the \texttt{\textdollar{}FILE} placeholder is replaced with the absolute path of the open file.

Services like the build command service do more than just read files. They also provide an API to write configuration objects into JSON files. This feature allows users to modify configuration data directly through Pie's interface, eliminating the need to manually edit files. The \texttt{BuildCommandService.WriteBuildCommandsToFile} method maps a \texttt{List<BuildCommand>} object to a serialized JSON structure. This structure will overwrite the data that relies in \texttt{config/build.json}.

\begin{lstlisting}[language=csharp, caption={Serializing build command configuration data into a JSON structure}]
public static void WriteBuildCommandsToFile(string file, List<BuildCommand> tempCommands)
{
    // Empty the configuration file
    File.WriteAllText(AppDomain.CurrentDomain.BaseDirectory + file, "");

    TextWriter textWriter = new StreamWriter(AppDomain.CurrentDomain.BaseDirectory + file);

    // Use Newtonsoft.Json's Writer
    using (JsonWriter writer = new JsonTextWriter(textWriter))
    {
        writer.Formatting = Formatting.Indented;

        writer.WriteStartArray();

        // Add every build command to a list of JSON items
        foreach (BuildCommand buildCommand in tempCommands)
        {
            writer.WriteStartObject();
            writer.WritePropertyName("name");
            writer.WriteValue(buildCommand.BuildCommandName);                    
            writer.WritePropertyName("command");
            writer.WriteValue(buildCommand.BuildCommandCmd);
            writer.WriteEndObject();
        }

        writer.WriteEndArray();
    }
}
\end{lstlisting}

Whenever users access the build commands, the objects loaded into the global variable are displayed in an \texttt{ObjectListView} control. Users can add, remove, or edit commands. Any modifications to the build commands are updated internally and, after the user initiates the saving, the changes are written back to the \texttt{config/build.json} file.

This is how Pie persists configuration data. While I have focused on the management logic for build commands, the same approach applies to database connections, Git credentials, themes, and additional settings like autocomplete and autosave. All of these configurations are stored as key-value pairs in JSON files, which are read and overwritten when users make changes to them.

\section{Theming Pie: Propagating custom colors to all controls}

As specified in the previous section, themes are processed in the same way as build commands. However, themes include more keys, and the values are comma-delimited strings of RGB (red, green, blue) colors. Here is a snippet from a JSON-formatted theme configuration file for Pie:

\begin{lstlisting}[language=json, caption={Theme configuration example in Pie}]
{
	"Primary": "35, 39, 56",
	"Secondary": "46, 52, 73",
	"Button": "67, 77, 107",
	"ButtonHover": "80, 91, 128",
	"Fore": "237, 238, 255",
	"FormBorder": "60, 64, 95",
	"Selection": "88, 94, 118",
	"CaretLineBack": "48, 54, 78",
	"NumberMargin": "36, 42, 63",
	"Folding": "46, 52, 73",
	"Comment": "63, 71, 102",
    "String": "235, 153, 87",
	"Operator": "250, 234, 110",
    "IconType": "light",
    (...)
}
\end{lstlisting}

Each theme has its own configuration file, stored in \texttt{config/themes/theme-name.json}. There are several predefined theme configurations available that can be used as templates for creating new themes. The theme configuration file specifies color schemes for all of Pie's controls, covering everything from the form (e.g. "Fore" and "FormBorder") to the syntax highlighting in Scintilla (e.g. "Comment", "String", and "Operator"). The currently selected theme is stored in the \texttt{config/theme.json} file and is updated both internally and externally whenever the user selects a different theme. The selected theme is kept in a global variable.

All form controls are part of the KryptonSuite and can have a \texttt{KryptonPallete} object attached to them. This global pallete instance applies the color scheme of the selected theme to various control colors, such as border color, background color, foreground color and hover color. The color processing is carried out in each form's constructor before the form is displayed on the screen. This ensures that there are no inconsistencies in the color mix of the application.

\begin{lstlisting}[language=csharp, caption={AboutForm's constructor, setting the global kryptonPalette object to each control visible in the form}]
public AboutForm()
{
    InitializeComponent();
    this.Palette = Globals.kryptonPalette;
    kryptonPanel1.Palette = Globals.kryptonPalette;
    kryptonLabel1.Palette = Globals.kryptonPalette;
    kryptonLabel2.Palette = Globals.kryptonPalette;
    kryptonLabel3.Palette = Globals.kryptonPalette;
    kryptonButton1.Palette = Globals.kryptonPalette;
    kryptonButton2.Palette = Globals.kryptonPalette;
}
\end{lstlisting}

The \texttt{ChangeTheme()} method explained below is triggered when the user picks a new theme from the top menu strip:

\begin{lstlisting}[language=csharp, caption={The ChangeTheme() method extracts the color scheme definition of a theme and sets its corresponding colors to Pie's global kryptonPalette control}]
public void ChangeTheme(string theme)
{
    // Explained separately
    ControlHelper.SuspendDrawing(this);

    Globals.theme = theme;

    // Explained separately
    Globals.colorDictionary = ThemeService.GetColorDictionary(theme);

    // The selected theme is persisted for when the user reopens the application
    ThemeService.WriteThemeToFile("config/theme.json", Globals.theme);

    // Find which tabs have a Scintilla editor and change their colors too
    for (int i = 0; i < tabControl.Pages.Count; i++)
    {
        if (Globals.tabInfos[i].getTabType() == TabType.CODE)
        {
            KryptonPage kryptonPage = tabControl.Pages[i];
            Scintilla scintilla = (Scintilla)kryptonPage.Controls[0];
            (...)

            if (Globals.tabInfos[i].getOpenedFilePath() != null)
            {
                // If a file is opened, also change the syntax highlighting color and the color of the autocomplete menu
                string extension = ParsingService.GetFileExtension(Globals.tabInfos[i].getOpenedFilePath());
                ColorizeTextArea(scintilla);
                ColorizeAutocompleteMenu(Globals.tabInfos[i].getAutocompleteMenu());
                (...)
            }
            else
            {
                // If no file is opened, only change the background number margin and code folding color of Scintilla
                ColorizeTextArea(scintilla);
            }
        }
    }

    // Also change the colors of the Palette
    ThemeService.SetPaletteToTheme(kryptonPalette, Globals.theme);
    (...)

    ControlHelper.ResumeDrawing(this);
    this.RedrawNonClient();
}
\end{lstlisting}

The \texttt{ControlHelper.SuspendDrawing()} and \texttt{ControlHelper.ResumeDrawing()} methods freeze the UI thread until all colors have been updated. This ensures that modifications are displayed consistently across all of Pie's components. 

At application start, all themes are read into memory along with their color scheme definitions, which are stored in a \texttt{Dictionary<string, Color>} map. The \texttt{ThemeService.GetColorDictionary()} method, mentioned in the snippet above, iterates over the internally saved themes and extracts the dictionary for the theme name provided as an argument. This dictionary is then saved as a global variable because it will be heavily used in the future. Iterating over an entire list of themes multiple times can significantly reduce runtime performance.

\begin{lstlisting}[language=csharp, caption={ThemeService.SetPaletteToTheme() maps a color dictionary to corresponding control components from a KryptonPalette control}]
public static void SetPaletteToTheme(KryptonPalette kryptonPalette, string theme)
{
    kryptonPalette.Common.StateCommon.Back.Color1 = Globals.colorDictionary["Secondary"];
    kryptonPalette.Common.StateCommon.Back.Color2 = Globals.colorDictionary["Secondary"];
    kryptonPalette.Common.StateCommon.Border.Color1 = Globals.colorDictionary["Secondary"];
    kryptonPalette.Common.StateCommon.Border.Color2 = Globals.colorDictionary["Secondary"];
    (...)
}
\end{lstlisting}

\section{Tab management in the code editor}

Pie displays different dialogs in tabs. The tab management logic is primarily handled by the \texttt{KryptonDockableNavigator} component, which is a wrapper around WinForms' native \texttt{TabControl}, offering similar functionality. Most of the tab functions can be accessed through the context menu (by right-clicking on a tab) or by using specific key combinations.

\begin{lstlisting}[language=csharp, caption={An event listener that listens for tab-related key bindings in Pie}]
private void keyDownEvents(object sender, KeyEventArgs e)
{
    (...)
    if (e.KeyCode == Keys.T && e.Modifiers == Keys.Control)
    {
        NewTab(TabType.CODE, null);
        e.SuppressKeyPress = true;
    }
    else if (e.KeyCode == Keys.W && e.Modifiers == Keys.Control)
    {
        CloseTab();
    }
    (...)
}
\end{lstlisting}

The \texttt{NewTab()} method creates a new \texttt{KryptonPage} object (Krypton's implementation of the \texttt{TabPage} from .NET) and requires a \texttt{TabType} enum as the first argument. The enum can be of type \texttt{CODE}, \texttt{GIT}, \texttt{RENDER\_HTML} or \texttt{RENDER\_MD}. The code snippet above creates a new tab of type \texttt{CODE}. The snippet below presents the implementation of \texttt{NewTab()}:

\begin{lstlisting}[language=csharp, caption={Implementation of the NewTab() method for CODE tabs}]
public void NewTab(TabType tabType, String path)
{
    // Create a new tab
    KryptonPage kryptonPage = new KryptonPage();
    kryptonPage.Text = "Untitled";
    kryptonPage.ToolTipTitle = kryptonPage.Text;

    // openedFilePath will only get initialized if the created tab is of RENDER\_* type
    string openedFilePath = null;
    AutocompleteMenu autocompleteMenu = null;

    if (tabType == TabType.CODE)
    {
        // Create a Scintilla editor and attach it to the tab page
        Scintilla TextArea = CreateNewTextArea();
        kryptonPage.Controls.Add(TextArea);
        // Initialize the autocomplete menu and attach it to the Scintilla editor
        autocompleteMenu = InitializeAutocompleteMenu(TextArea);
    }
    (...)

    // Insert the tab next to the currently selected tab
    int index = tabControl.Pages.Count <= 0 ? 0 : tabControl.SelectedIndex+1;

    tabControl.Pages.Insert(tabControl.SelectedIndex+1, kryptonPage);
    tabControl.SelectedPage = kryptonPage;

    // Create a new TabInfo object and add it to the list of tabs
    TabInfo tabInfo = new TabInfo(openedFilePath, false, tabType, autocompleteMenu);
    Globals.tabInfos.Insert(index, tabInfo);
    
    (...)
}
\end{lstlisting}

The \texttt{TabInfo} object holds metadata about Pie's opened tabs, and this data is accessed by different parts of the code. Each time a new tab is created, a new \texttt{TabInfo} instance is appended to a list at the same \texttt{i} index as the tab in the tab manager.

\begin{lstlisting}[language=csharp, caption={Structure of the TabInfo class}]
public class TabInfo
{
    // CODE, GIT, RENDER_MD or RENDER_HTML
    private TabType tabType;

    // Instance of the Autocomplete object, if the tab is of type CODE
    private AutocompleteMenu autocompleteMenu;

    // Flags for CODE tabs
    private bool openedFileChanges;
    private string openedFilePath;

    // Getters and setters
    (...)
}
\end{lstlisting}

The \texttt{openedFileChanges} flag is set to \texttt{true} for each tab of type \texttt{CODE} whenever the user inserts or removes a character from the original file. If the file is saved, the flag is set to \texttt{false}. When a tab is closed, the logic first checks if \texttt{openedFileChanges} is \texttt{true} and, if so, asks the user if they want to save their changes.

The \texttt{openedFilePath} will be updated if the user saves the file to a specific path (using Ctrl + S or navigating to "File" and choosing "Save"). The file path is stored internally to analyze aspects such as the file extension and update Scintilla's syntax highlighting for that tab. The snippet below presents the implementation of the \texttt{Save()} method, which uses the tab metadata in its logic:

\begin{lstlisting}[language=csharp, caption={Pie's implementation of the Save() method that gets called whenever the user wants to save a file inside a CODE tab}]
public void Save(int openedTabIndex)
{
    if (Globals.tabInfos[openedTabIndex].getOpenedFilePath() == null)
    {
        // If no file has yet been saved initially, let the user pick a path for their file. SaveAs() opens a SaveFileDialog before updating tab metadata and then writes to the output file
        SaveAs(openedTabIndex);
    }
    else
    {
        // A path is already kept in memory on the i-th position. The file should be saved in the same place
        string chosenPath = Globals.tabInfos[openedTabIndex].getOpenedFilePath();

        // Save the file
        TextWriter txt = new StreamWriter(chosenPath);
        Scintilla TextArea = (Scintilla)tabControl.Pages[openedTabIndex].Controls[0];
        txt.Write(TextArea.Text);
        txt.Close();

        // Changed the title of the saved tab from "Untitled" to the name of the file
        tabControl.Pages[openedTabIndex].Text = ParsingService.GetFileName(chosenPath);

        // Get the extension of the saved file and update Scintilla's lexer accordingly
        string fileName = ParsingService.GetFileName(Globals.tabInfos[openedTabIndex].getOpenedFilePath());
        string extension = ParsingService.GetFileExtension(fileName);
        ScintillaLexerService.SetLexer(extension, TextArea, tabControl, openedTabIndex);
    }

    // Additional menu item toggling logic goes here
    (...)
}
\end{lstlisting}

\section{Syntax highlighting in ScintillaNET}

ScintillaNET provides several predefined lexers that allow stylization of language-specific tokens such as reserved keywords, braces and operators. The \texttt{LexerService} class offers syntax highlighting for various languages by combining predefined lexers with a list of custom keywords. For example, while ScintillaNET does not include a default style for Java, it does support C++. The syntax for Java and C++ is very similar: both use braces to define an inner scope, they share the same syntax for if statements, and their operators are quite alike. The primary difference lies in their keywords. Thus, the Java language support in Pie uses Scintilla's C++ lexer along with a list of reserved keywords extracted from Oracle's Java documentation.

\begin{lstlisting}[language=csharp, caption={The ConfigureLexer() method that is called whenever the user opens a new file or changes Pie's theme}]
public static void ConfigureLexer(string language, Scintilla scintilla, KryptonDockableNavigator tabControl, int index)
{
    (...)
    
    if (language == "java")
    {
        // Define token highlighting for C++
        scintilla.Lexer = Lexer.Cpp;

        // Map current color dictionary to Scintilla's highlighting
        scintilla.Styles[ScintillaNET.Style.Cpp.Default].ForeColor = parserColorDictionary["Default"];
        scintilla.Styles[ScintillaNET.Style.Cpp.Comment].ForeColor = parserColorDictionary["Comment"];
        scintilla.Styles[ScintillaNET.Style.Cpp.Number].ForeColor = parserColorDictionary["Number"];
        scintilla.Styles[ScintillaNET.Style.Cpp.Word].ForeColor = parserColorDictionary["Word"];
        scintilla.Styles[ScintillaNET.Style.Cpp.Word2].ForeColor = parserColorDictionary["Word2"];
        scintilla.Styles[ScintillaNET.Style.Cpp.String].ForeColor = parserColorDictionary["String"];
        scintilla.Styles[ScintillaNET.Style.Cpp.Operator].ForeColor = parserColorDictionary["Operator"];
        scintilla.Styles[ScintillaNET.Style.Cpp.Preprocessor].ForeColor = parserColorDictionary["Preprocessor"];
        
        (...)

        // Add Java keywords
        string keywordSet1 = "abstract break case catch continue default do else extern false finally for native super extends final native transient volatile implements synchronized if instanceof import package interface new null private protected public record return sizeof switch this throw throws true try while";
        string keywordSet2 = "boolean Boolean byte Byte char Character class double Double enum float Float int Integer long Long short Short static String void";

        scintilla.SetKeywords(0, keywordSet1);
        scintilla.SetKeywords(1, keywordSet2);

        // Also initialize the autocomplete menu with the list of keywords
        string combined = keywordSet1 + " " + keywordSet2;
        string[] combinedArray = combined.Split(' ');
        SetAutocompleteMenuKeywords(Globals.tabInfos[index].getAutocompleteMenu(), combinedArray.ToList());

        EnableFolding(scintilla);
    }

    (...)
}
\end{lstlisting}

Two color definitions exist for reserved keywords. Statements (\texttt{keywordSet1}) and variable types (\texttt{keywordSet2}) have different colors specified by the \texttt{Word} and \texttt{Word2} keys in the theme configuration. This has been implemented to improve code comprehension.

Defining a new language in Pie requires modifyingthe \texttt{ConfigureLexer()} method to add additional support. Both a lexer and a list of keywords need to be provided. Keywords are typically specified in the language's official documentation. An existing lexer can also be used for languages that share similarities with others (such as the C, C++, and C\# suite). However, languages such as COBOL or Lisp require a new lexer with custom parsing logic. Implememnting such a lexer can take hours or even days for a single programmer.

\section{Initiating database connections and sending queries}

Similar to the build command processing logic, database connections are also stored in JSON files. The code snippet below shows an example of a configuration file containing metadata about one database connection:

\begin{lstlisting}[language=json, caption={Structure of the JSON-formatted database connections file in Pie}]
[
  {
    "connectionName": "Local MySQL Database",
    "databaseType": "MySQL",
    "hostname": "localhost",
    "port": 3306,
    "databaseName": "mydb",
    "username": "root",
    "password": "admin"
  }
]
\end{lstlisting}

The \texttt{connectionName} parameter represents a custom name that makes the database connection easier to identify in the list. The \texttt{databaseType} parameter determines the driver that will be used when initializing the connection with the server. The other parameters are directly used in the database connection string.

Connectivity is managed through the \texttt{DatabaseService} class. Users can check if the connection has been successfully established before querying tables within the database. The \texttt{CheckDatabaseConnection()} method, shown in the following code snippet, maps a database type to its corresponding driver, creates the connection string, and calls the \texttt{AttemptConnection} method, which attempts to establish a connection with the server. The abstracted way of implementing database connectors in C\# provides good extensibility, as all three driver types implement the \texttt{IDbConnection} interface.

\begin{lstlisting}[language=csharp, caption={The CheckDatabaseConnection() method that tries to initiate a connection with a database through its parameters}]
public static DatabaseResponse CheckDatabaseConnection(DatabaseType databaseType, string hostname, int port, string databaseName,  string username, string password)
{
    if (databaseType == DatabaseType.MySQL)
    {
        // Build connection string for MySQL database
        string myConnectionString = "server=" + hostname + ";port=" + port + ";database=" + databaseName + ";uid=" + username + ";pwd=" + password + ";";

        // Try to connect
        return AttemptConnection(new MySqlConnection(myConnectionString));
    }
    else if (databaseType == DatabaseType.MSSQL)
    {
        // Build connection string for MSSQL database
        SqlConnectionStringBuilder connectionStringBuilder = new SqlConnectionStringBuilder()
        {
            DataSource = hostname + "," + port,
            InitialCatalog = databaseName,
            UserID = username,
            Password = password
        };

        // Try to connect
        return AttemptConnection(new SqlConnection(connectionStringBuilder.ConnectionString));
    }
    else if (databaseType == DatabaseType.PostgreSQL)
    {
        // Build connection string for PostgreSQL database
        string strConnString = "Server=" + hostname + ";Port=" + port + ";User Id=" + username + ";Password=" + password + ";Database=" + databaseName;

        // Try to connect
        return AttemptConnection(new NpgsqlConnection(strConnString));
    }

    return null;
}

private static DatabaseResponse AttemptConnection(IDbConnection connection)
{
    try
    {
        // Try to initiate the connection
        connection.Open();
        connection.Close();

        // If no exception was thrown, return successful status
        return new DatabaseResponse(true, null);
    }
    catch (DbException ex)
    {
        return new DatabaseResponse(false, ex.InnerException.Message);
    }
    catch (InvalidOperationException ex)
    {
        return new DatabaseResponse(false, ex.Message);
    }
}
\end{lstlisting}

The returned \texttt{DatabaseResponse} object provides data and metadata about the database interaction process. \texttt{Success} is set to \texttt{true} if no exceptions were thrown during the interaction with the database. \texttt{Message} will contain more details about the exception if any error occured. The \texttt{DataTable} object will be used when querying databases and contains the output of the SQL query sent to the server.

\begin{lstlisting}[language=csharp, caption={The DatabaseResponse class}]
public class DatabaseResponse
{
    public bool Success { get; private set; }
    public string Message { get; private set; }
    public DataTable DataTable { get; private set; }

    public DatabaseResponse(bool success, string message)
    {
        Success = success;
        Message = message;
    }

    public DatabaseResponse(bool success, string message, DataTable dataTable)
    {
        Success = success;
        Message = message;
        DataTable = dataTable;
    }
}
\end{lstlisting}

Running queries on a database is a similar process, but in addition to returning connection metadata, the service also returns the execution output in a \texttt{DataTable} object.

\begin{lstlisting}[language=csharp, caption={The ExecuteSQLCommand() that sends a query to a database}]
public static DatabaseResponse ExecuteSQLCommand(string query, DatabaseConnection databaseConnection)
{
    DataTable dt = new DataTable();

    if (databaseConnection.DatabaseType == DatabaseType.MySQL)
    {
        // Build connection string for MySQL database
        string myConnectionString = "server=" + databaseConnection.Hostname + ";port=" + databaseConnection.Port + ";database=" + databaseConnection.DatabaseName + ";uid=" + databaseConnection.Username + ";pwd=" + databaseConnection.Password + ";";

        MySqlConnection cnn = new MySqlConnection(myConnectionString);
        try
        {
            // Create the connection. A exception will be thrown if the connection cannot be established successfully 
            cnn.Open();

            // Create the command and execute the query
            MySqlCommand cmd = new MySqlCommand(query, cnn);
            MySqlDataAdapter sda = new MySqlDataAdapter(cmd);
            sda.Fill(dt);

            // Close the connection
            cnn.Close();

            // No rows were returned (this can happen if the user runs a query different from "SELECT")
            if (dt.Columns.Count == 0)
            {
                return new DatabaseResponse(false, "No columns returned.", null);
            }

            // Rows were returned. Add the DataTable object to the response
            return new DatabaseResponse(true, null, dt);
        }
        catch (DbException ex)
        {
            if (ex.InnerException != null)
            {
                return new DatabaseResponse(false, ex.InnerException.Message, null);
            }
            else
            {
                return new DatabaseResponse(false, ex.Message, null);
            }
        }
    }

    (...)
}
\end{lstlisting}

After executing a query, a control of type \texttt{DatabaseOutputForm} will be displayed, containing a \texttt{KryptonDataGridView} element. The \texttt{DataTable} object returned through the \texttt{DatabaseResponse} will be set as a source of the grid view, allowing users to see the output in a formatted table, similar to modern database visualizers.

\section{Advanced VCS management with LibGit2Sharp}

Git repository management is handled from a separate tab within the main form. Users can select the path of their local repository, which is then stored in a global variable and initialized using the LibGit2Sharp library.

\begin{lstlisting}[language=csharp, caption={Using LibGit2Sharp to scan a local repository provided as a path}]
Globals.repo = new Repository(path);
\end{lstlisting}

The \texttt{RetrieveGitItemsForCurrentBranch()} method is called when the user opens a repository for the first time or refreshes the status of the currently opened repository.

\begin{lstlisting}[language=csharp, caption={The RetrieveGitItemsForCurrentBranch() method that retrieves the status of the files located inside a repository and displays them in an ObjectListView control}]
private void RetrieveGitItemsForCurrentBranch()
{
    StatusOptions statusOptions = new StatusOptions();
    statusOptions.IncludeUnaltered = true;

    // Will be used for displaying file info in an ObjectListView
    List<GitFile> gitFileList = new List<GitFile>();

    // Use LibGit2Sharp to retrieve file status for all files in the repository
    foreach (var item in Globals.repo.RetrieveStatus(statusOptions))
    {
        // Now map each file status to a GitFile element
        GitFile gitFile = new GitFile();
        gitFile.Name = item.FilePath;

        if (item.State == FileStatus.DeletedFromWorkdir)
        {
            gitFile.Status = "Deleted";
        }
        else if (item.State == FileStatus.Ignored)
        {
            gitFile.Status = "Ignored";
        }
        else if (item.State == FileStatus.ModifiedInIndex || item.State == FileStatus.ModifiedInWorkdir)
        {
            gitFile.Status = "Modified";
        }
        else if (item.State == FileStatus.NewInIndex || item.State == FileStatus.NewInWorkdir)
        {
            gitFile.Status = "New";
        }
        else
        {
            gitFile.Status = item.State.ToString();
        }

        gitFileList.Add(gitFile);
    }

    // Add the metadata to the ObjectListView located in the Git tab
    gitStagingAreaListView.SetObjects(gitFileList);
}
\end{lstlisting}

Commiting items to the repository is also done through LibGit2Sharp's API. Additionally, I have provided several validations to ensure that the user performs the intended actions. Before commiting, Pie checks if the repository's status is "dirty", meaning that files have been changed in the staging area. If this is true, the application proceeds with staging the files. It then checks if commit credentials have been provided; if not, the user is prompted to provide them. LibGit2Sharp then handles the process by creating a signature with the user's information and initiating the commit.

\begin{lstlisting}[language=csharp, caption={Commiting to a repository using LibGit2Sharp}]
private void GitCommit(string items)
{
    if (Globals.repo != null)
    {
        RepositoryStatus status = Globals.repo.RetrieveStatus();

        // If there are modified files in the staging area, proceed with the commit
        if (status.IsDirty)
        {
            // If not all files have been selected, call another overloaded version of Stage()
            if (items != "*")
            {
                List<string> files = items.Split(' ').ToList();

                Commands.Stage(Globals.repo, files);
            }
            else
            {
                // Simply commit all files
                Commands.Stage(Globals.repo, items);
            }

            // If no credentials have been provided, prompt the user to type in their name and email address
            if (string.IsNullOrEmpty(Globals.gitCredentials.Name) || string.IsNullOrEmpty(Globals.gitCredentials.Email))
            {
                GitCommitCredentialsForm gitCredentialsForm = new GitCommitCredentialsForm();
                Globals.gitFormClosedWithOk = false;
                gitCredentialsForm.ShowDialog();

                if (Globals.gitFormClosedWithOk)
                {
                    GitService.WriteCredentials(Globals.gitCredentials);

                    // Retry the commit
                    GitCommit(items);
                }
            }
            else
            {
                // Create a signature based on the user's credentials and run the commit
                Signature signature = new Signature(Globals.gitCredentials.Name, Globals.gitCredentials.Email, DateTime.Now);

                string commitText = commitMessageRichTextBox.Text;

                Task.Run(() =>
                {
                    Globals.repo.Commit(commitText, signature, signature);
                }).Wait();

                (...)
                
                ShowNotification("Successfully commited.");
            }
        }
        else
        {
            ShowNotification("You have nothing to commit.");
        }
    }
}
\end{lstlisting}

\section{Text formatting functionality}

When the user clicks on the "Format" button located on Pie's top menu strip, a \texttt{FormatForm} pops up, allowing the user to search through various formatting techniques. Double-clicking on one of the options will close the form and launch the format job on the currently selected tab (which must be of type \texttt{CODE}). The formatting logic is provided through the \texttt{FormattingService} class.

The content of the Scintilla editor is sent to one of the service's methods, which perform one of the following actions:

\begin{enumerate}
  \item process text line by line;
  \item process text character by character;
  \item apply sorting algorithms.
\end{enumerate}

A commonly used line processing technique is the one that removes newline characters ('\n') and adds commas (',') at the end of each line. This is helpful if, for example, an output generated by a table processing product like Microsoft Excel needs to be converted to a Python list.

\begin{lstlisting}[language=csharp, caption={Converting newline to commas using the ConvertNewlineToComma() method from the FormattingService class}]
public static string ConvertNewlineToComma(string text)
{
    string[] lines = text.Split('\n');

    for (int i = 0; i<lines.Length; i++)
    {
        if (lines[i].Length-1 >= 0 && lines[i][lines[i].Length-1] == '\r')
        {
            lines[i] = lines[i].Substring(0, lines[i].Length-1);
        }
    }

    return lines.Aggregate((curr, next) => {
        return curr + "," + next;
    });
}
\end{lstlisting}

Character processing algorithms primarily focus on trimming whitespaces and sanitizing text inputs. The \texttt{RemoveAllConsecutiveWhitespaces()} method removes extra space characters from the Scintilla editor.

\begin{lstlisting}[language=csharp, caption={Removing consecutive whitespaces through the RemoveAllConsecutiveWhitespaces() method from the FormattingService class}]
public static string RemoveAllConsecutiveWhitespaces(string text)
{
    string result = "";

    for (int i = 0; i < text.Length; i++)
    {
        if (char.IsWhiteSpace(text[i]))
        {
            if (result.Length == 0 || !char.IsWhiteSpace(result[result.Length - 1]) || text[i] == '\n' || text[i] == '\r')
            {
                result += text[i];
            }
        }
        else
        {
            result += text[i];
        }
    }

    return result;
}
\end{lstlisting}

Sorting lines is also a common formatting technique. It allows users to order names and other strings in ascending or descending order.

\begin{lstlisting}[language=csharp, caption={Sorting lines ascending or descending through the SortLines() method from the FormattingService class}]
public static string SortLines (string text, bool ascending)
{
    if (text.Equals(""))
    {
        return "";
    }

    string result = "";

    List<string> lines = text.Split('\n').ToList();

    lines.Sort();

    if (!ascending)
    {
        lines.Reverse();
    }

    for(int i = 0; i<lines.Count; i++)
    {
        if (i == lines.Count - 1)
        {
            result += lines[i];
        }
        else
        {
            result += lines[i] + "\n";
        }
    }

    if (result[result.Length - 1] == '\r')
    {
        return result.Substring(0, result.Length - 1);
    }
    else
    {
        return result;
    }

    return result;
}
\end{lstlisting}

The methods will return a string, which will then overwrite the content of the selected Scintilla editor.