\thispagestyle{pagestyle}

\begin{center}
    \textbf{\fontsize{20pt}{24pt} \selectfont Abstract}
\end{center}

Lumea dezvoltatorilor are nevoie de un editor de cod care să ofere o modalitate mai eficientă de scriere a programelor, fără ca utilizatorii să fie nevoiți să apeleze la configurații adiționale sau unelte externe. Proiectele mici nu necesită un mediu de dezvoltare (IDE) cu exces de funcționalități, acestea putând crește considerabil complexitatea procesului de realizare al aplicațiilor software. Un mediu bun de dezvoltare pentru asfel de cazuri trebuie să ofere minimul necesar scrierii de cod într-o manieră decentă, acest lucru fiind însă foarte greu de obținut în ziua de azi, când programatorii se folosesc de tot mai multe unelte pentru a își spori productivitatea.

Gama variată de tehnologii și framework-uri nu ajută nici ea. Programatorii din ziua de azi nu se bazează pe un singur stack de dezvoltare când construiesc o aplicație. Așadar, un editor de cod trebuie să suporte mai multe limbaje de programare și să fie constant îmbunătățit și extins.

Realizarea unui editor de cod de uz general care să fie și ușor de folosit este o treabă dificilă, ce necesită analiza detaliată a uneltelor interne și externe dintr-un IDE, prioritizându-le după cât de folosite sunt. Ținând cont de aceste aspecte, am construit Pie. Pie este un editor de cod de uz general ce oferă câteva din cele mai utilizate funcționalități ale mediilor de dezvoltare integrate, accesibile dintr-o interfață de utilizator simplificată. Acest tool le oferă programatorilor posibilitatea de a se concentra pe cod, în loc să petreacă timp încercând combinații de taste sau citind documentație. Întreaga analiză de funcționalități a fost efectuată de mine, timp în care am folosit Pie ca un înlocuitor complet pentru alte editoare.

Pie, scris integral în C\#, a fost construit folosind tehnologia Windows Forms ("WinForms") a Microsoft, și este intenționat să înlocuiască în totalitate mediile de dezvoltare ca Visual Studio Code sau IntelliJ, atunci când functionalitățile acestora devin deja prea mult pentru anumite task-uri simple. Proiectul a fost gândit și ca un potențial înlocuitor pentru editoare de text mai mici, ca și Notepad++, având integrate și câteva funcționalități de manipulare a textului.

Demo-ul prezentat în această lucrare va prezenta interfața intuitivă a lui Pie, explicând câteva cazuri de utilizare ale aplicației. Se va efectua și o comparație între Pie și cele mai populare editoare de cod din prezent, scoțând în evidență ceea ce oferă produsul meu.  

\vfill