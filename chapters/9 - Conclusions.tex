\chapter{Conclusions}
\thispagestyle{pagestyle}

In this work, I have presented a method to enhance productivity in the software development field by introducing an application suitable for small-scale projects. In "State of the Art", I explored the history of code editors, from the punch card era to the present day. I also performed a comparison between my solution, Pie, and the most popular integrated development environments on the market. This comparison demonstrates that Pie not only includes all the essential features of an IDE but also offers an intuitive user interface that simplifies access to these features.

In the next chapter, "Technologies", I discussed the technologies that form Pie's building blocks, with a particular focus on Microsoft's Windows Forms, a mature user interface solution for the .NET framework. The "Application Structure" chapter introduced Pie's "components" and showcased several features of the editor, accompanied by screenshots.

A way of implementing the application's features using the C\# language was discussed in the "Implementation" chapter, where I also included code snippets explained in comments and additional paragraphs. Finally, the "Usage Scenarios" chapter illustrated the intended use cases for Pie.

Pie has been well received by the developer community, and an article about it is set to be published at the SACI (IEEE 18th International Symposium on Applied Computational Intelligence) 2024 conference in Timișoara. By staying in close contact with developers from Romania through various programming communities and forums, I have received valuable feedback that helped me take important decisions regarding future integrations for my product.

As future work, I plan on launching the application in a beta version in order to gather more feedback from its users. I consider that the best feedback is the one received by users actually testing the solution, not just reading about it. Gathering feedback will allow me to refine Pie's present features and also add additional ones, if needed.

I also plan to improve Pie's Git interface, as it does not provide all the features found in standalone repository explorers. Pie should also provide users the option to roll back changes in a file, inspect every change through the Git log, and perform branch operations like merge or rebase. The database manager is also set for an update, as it will support more relational database types and have better error handling.